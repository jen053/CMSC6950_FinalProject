\documentclass{article}
\usepackage[utf8]{inputenc}
\usepackage{graphicx}
\usepackage{subcaption}
\usepackage{geometry}
\geometry{left=38mm,top=30mm,bottom=30mm,right=25mm}

\begin{document}
\begin{titlepage}
\newcommand{\HRule}{\rule{\linewidth}{0.5mm}}

\center
\textsc{\LARGE \textbf{Open Science Project}}\\[1 cm]

\textsc{\Large Windrose Python Package}\\[0.5 cm]

\textsc{\large Jacob Newman (201528601)}\\[0.5 cm]





\vfill\vfill\vfill
{\large\today}
\vfill

\end{titlepage}


\section{Introduction}\label{Introduction}
Great strives to better understand and display historical climatological data have been made in the past decade due to growing concerns over climate change among other things. One climatological data set 
that has a diverse range of importance when addressing these problems is wind data. Changes in major wind patterns, such as increases in the power of wind systems over time, can be used as a gauge to 
measure how the strength of storms is changing due to climate change (Mendelsohn et al. 2012). Another application of study of wind data, in particular directional wind data, is designing wind power 
farms (Cetinay, Kuipers, and Guven, 2016) for a source of clean, renewable energy. For optimal placement of wind turbines, promenent wind directions and the associated wind speeds need to be plotted and 
analyzed. Other applications of studying wind data include mapping air pollution sources (Adams and Kanaroglou, 2016) and knowledge of dominate wind directions at airports, important information for 
pilots and air traffic controllers (Bellasio, 2014).
\\
\indent Here we will look at the last of the applications in connection to airports in Newfoundland and Labrador. Wind speed and directional data are collected at each airport through out the province, 
making data access relatively simple. A Python based wind data visulization package Windrose (Roubeyrie and Celles, 2018) will be employed to show several variations of polar diagrams used to plot each 
airport's respective wind data. Although we propose the purpose of this study to be that of knowledge gathering for pilots and air traffic controllers, a broad connection to the potential for wind 
energy production in the province of Newfoundland and Labrador (a long discussed topic) can be made. We a merely limited by the availablity of data to fully engage in the study of wind power potential. 
\\
\indent The remainder of this report is structed as follows, in section (\ref{Data_access}) we discuess data access and collection, in section (\ref{Wind_data_visualization}) we introduce the Windrose  
package and show multiple visualizations of direction wind data for a user-defined airport (see README), finally in section (\ref{Discussion_and_conclusions}) we breifly discuss and conclude on some key 
points.      

\section{Methodology}\label{Methodology}

\subsection{Data Access}\label{Data_access}
The README file should be consluted for prelimnanry data access information and for instructions on how to call data from a particular airport. Here we talk about where the data is from and how it is 
collected. All the data (fithteen sets) available from the windData repository was accessed through the Canadian Climate Data Accessibility Portal (CCDAP). The CCDAP is a public data query platform which 
holds Canadian historical climate data collected and maintained by Environment and Climate Change Canada. CCDAP was made at the Water Security and Climate Change Lab (WSCC) at Concordia University. As 
mentioned above, each data set was collected at an airport in Newfoundland and Labrador over some range of time (given in data set name). The data has been stripped to only include data/time, wind speed 
and wind direction information; full data sets for each locality include multiple addtional weather measures such as temperature and visiablity. The decision to only include wind data in this report (as 
other data, visiablity in particular, is also important information for air travel) was for two main reasons, firstly, the CCDAP limits the size of data sets and thus including additional weather 
measurements would restricted the range of time of each data set subsequently degraded the historical aspect of the information gather by this study. Secondy, the Windrose package currently provides no 
mechinism for analizing weather data other then wind data.
\indent Each

\begin{figure}[h!]
\centering
\includegraphics[width=10cm]{Images/AWOS.jpg}
\label{AWOS}
\caption{Automated Weather Observation System (AWOS) used to monitor weather at many airports throughout Canada. Older data sets or data sets from remote areas would have been collected on a similar, 
less-advanced weather station.}
\end{figure}

\subsection{Wind Data Visualization}\label{Wind_data_visualization}

\begin{figure}[h!]
\begin{subfigure}{.55\textwidth}
\centering
\includegraphics[width=.8\linewidth]{bar.pdf}
\label{bar_windrose}
\end{subfigure}
\begin{subfigure}{.55\textwidth}
\centering
\includegraphics[width=.8\linewidth]{contour.pdf}
\label{contour}
\end{subfigure}
\begin{subfigure}{.55\textwidth}
\centering
\includegraphics[width=.8\linewidth]{contourf.pdf}
\label{contourf_windrose}
\end{subfigure}
\begin{subfigure}{.55\textwidth}
\centering
\includegraphics[width=.8\linewidth]{box.pdf}
\label{box_windrose}
\end{subfigure}
\label{windrose_diagrams}
\caption{Directional wind data plotted in different type of polar diagrams for the Windrose Python package. Upper left: bar; upper right: contour, lower left: filled contour; lower right: box.}
\end{figure}

\begin{figure}[h!]
\centering
\includegraphics[width=10 cm]{location.pdf}
\label{location}
\caption{Map of Newfoundland and Labrador annoted by location of wind data collection (yellow star).}
\end{figure}

\section{Disscusion and Conclusions}\label{Disscusion_and_conclusions}

\end{document}
