\documentclass{article}
\usepackage[utf8]{inputenc}
\usepackage{graphicx}


\begin{document}
\begin{titlepage}
\newcommand{\HRule}{\rule{\linewidth}{0.5mm}}

\center
\textsc{\LARGE \textbf{Open Science Project}}\\[1 cm]

\textsc{\Large Windrose Python Package}\\[0.5 cm]

\textsc{\large Jacob Newman (201528601)}\\[0.5 cm]





\vfill\vfill\vfill
{\large\today}
\vfill

\end{titlepage}


\section{Introduction}\label{Introduction}
Great strives to better understand and display historical climatological data have been made in the past decade due to growing concerns over climate change. Many packages have been developed to model 
future climate data and display previously collected data, an example of the former is the Windrose Python package (Roubeyrie et al., 2018). A wind rose is used to display wind speed and directional data 
for a particular location.  

\begin{figure}[h!]
\includegraphics[scale=0.6]{c:/Users/Jacob/CMSC6950_py/Open_Science_Project/CMSC6950_FinalProject/"St.John's.pdf"}
\label{St._John's_Windrose}
\caption{St. John's Windrose.}
\end{figure}

\end{document}
