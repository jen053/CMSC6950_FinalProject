\documentclass{article}
\usepackage[utf8]{inputenc}
\usepackage{graphicx}
\usepackage{subcaption}
\usepackage{geometry}
\geometry{left=38mm,top=30mm,bottom=30mm,right=25mm}

\begin{document}
\begin{titlepage}
\newcommand{\HRule}{\rule{\linewidth}{0.5mm}}

\center
\textsc{\LARGE \textbf{Open Science Project}}\\[1 cm]

\textsc{\Large Windrose Python Package}\\[0.5 cm]

\textsc{\large Jacob Newman (201528601)}\\[0.5 cm]





\vfill\vfill\vfill
{\large\today}
\vfill

\end{titlepage}


\section{Introduction}\label{Introduction}
Great strives to better understand and display historical climatological data have been made in the past decade due to growing concerns over climate change among other things. One climatological data set 
that has a diverse range of importance when addressing these problems is wind data. Changes in major wind patterns, such as increases in the power of wind systems over time, can be used as a gauge to 
measure how the strength of storms is changing due to climate change. Another application of study of wind data, in particular directional wind data, is designing wind power farms for a source of clean, 
renewable energy.   

\begin{figure}[h!]
\begin{subfigure}{.55\textwidth}
\centering
\includegraphics[width=.8\linewidth]{bar.pdf}
\label{bar_windrose}
\end{subfigure}
\begin{subfigure}{.55\textwidth}
\centering
\includegraphics[width=.8\linewidth]{contour.pdf}
\label{contour}
\end{subfigure}
\begin{subfigure}{.55\textwidth}
\centering
\includegraphics[width=.8\linewidth]{contourf.pdf}
\label{contourf_windrose}
\end{subfigure}
\begin{subfigure}{.55\textwidth}
\centering
\includegraphics[width=.8\linewidth]{box.pdf}
\label{box_windrose}
\end{subfigure}
\label{windrose_diagrams}
\caption{Directional wind data plotted in different type of polar diagrams for the Windrose Python package. 
Upper left: bar; upper right: contour, lower left: filled contour; lower right: box.}
\end{figure}

\begin{figure}[h!]
\centering
\includegraphics[width=\textwidth]{location.pdf}
\label{location}
\caption{.}
\end{figure}

\end{document}
